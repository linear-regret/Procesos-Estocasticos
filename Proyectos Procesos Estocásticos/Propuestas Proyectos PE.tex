\documentclass[10pt]{article}
\usepackage[spanish,es-tabla]{babel}
\usepackage[left=2.5cm,bottom=4.5cm]{geometry}
\usepackage{amsmath,amssymb}
\spanishdecimal{.}
\usepackage[utf8]{inputenc}
\usepackage{graphicx}
\usepackage{hyperref} % Para que las ecuaciones referencíen con links
\usepackage{booktabs}
\usepackage{fullpage}
\usepackage{float}
\usepackage{lscape}
\usepackage{parskip}
\setlength{\parindent}{0pt}
\title{Propuestas para proyectos}
\date{\today}
\author{Profesor: Rafael Miranda Cordero\\Ayudante:Fernando Avitúa Varela}
\begin{document}
\maketitle

\section*{Modelo de crecimiento tumoral con cadenas de Markov}

Este proyecto consiste en modelar el crecimiento de un tumor canceroso como un proceso de ramificación de Markov, donde cada célula puede dividirse, morir o permanecer inactiva con ciertas probabilidades. El objetivo es estimar el tamaño del tumor y su evolución en el tiempo, así como el efecto de posibles tratamientos. Una forma de mejorar el proyecto es incorporar más factores biológicos que afecten el crecimiento tumoral, como el suministro de oxígeno, la angiogénesis o la heterogeneidad celular.

\section*{Modelo de siniestralidad con procesos de Poisson}

Este proyecto consiste en modelar el número y la severidad de los siniestros que ocurren en una cartera de seguros como procesos de Poisson. El objetivo sería calcular la prima de riesgo y el capital de solvencia que debe tener una compañía de seguros para cubrir los posibles siniestros. Una forma de mejorar el proyecto es considerar diferentes distribuciones de probabilidad para la severidad de los siniestros, así como la dependencia entre los siniestros de diferentes tipos o categorías.

\section*{Modelo de predicción de series temporales con martingalas}

Este proyecto consiste en modelar una serie temporal, como el precio de una acción, el tipo de cambio o el clima, como una martingala, que es un proceso estocástico que tiene la propiedad de que el valor esperado de la siguiente observación es igual al valor actual. El objetivo es predecir el comportamiento futuro de la serie temporal y evaluar la precisión de las predicciones. Una forma de mejorar el proyecto es comparar el modelo de martingala con otros modelos de predicción, como el modelo autorregresivo, el modelo de media móvil o el modelo ARIMA.

\section*{Modelo de aprendizaje por refuerzo con movimiento browniano}

Este proyecto consiste en modelar el aprendizaje de un agente inteligente que interactúa con un entorno incierto y dinámico como un movimiento browniano, que es un proceso estocástico que describe el movimiento aleatorio de una partícula. El objetivo es encontrar la política óptima que maximice la recompensa acumulada del agente a largo plazo. Una forma de mejorar el proyecto es explorar diferentes algoritmos de aprendizaje por refuerzo, como el método de Monte Carlo, el método de diferencias temporales o el método de Q-learning.



\end{document}